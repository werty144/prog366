\documentclass[12pt]{article} % 12 -- размер шрифта
\usepackage{cmap} % Чтобы можно было копировать русский текст из pdf
\usepackage[T2A]{fontenc}
\usepackage[russian]{babel} % В частности эта строка отвечает за правильные переносы слов в конце строки
\usepackage[utf8]{inputenc} % Проверьте, что кодировка файла -- тоже utf8
\usepackage{amsmath, amssymb} % Чтобы юзать математические символы
\usepackage{ dsfont }
\usepackage{ wasysym }
\usepackage[makeroom]{cancel}
\usepackage{hyperref}
\usepackage{listings}

\usepackage[utf8]{inputenc}

\usepackage{listings}
\usepackage{xcolor}
\usepackage{etoolbox}

\definecolor{codegreen}{rgb}{0,0.6,0}
\definecolor{codegray}{rgb}{0.5,0.5,0.5}
\definecolor{codepurple}{rgb}{0.58,0,0.82}
\definecolor{backcolour}{rgb}{0.95,0.95,0.92}

\lstdefinestyle{mystyle}{
	backgroundcolor=\color{backcolour},   
	commentstyle=\color{codegreen},
	keywordstyle=\color{magenta},
	numberstyle=\tiny\color{codegray},
	stringstyle=\color{codepurple},
	basicstyle=\ttfamily\footnotesize,
	breakatwhitespace=false,         
	breaklines=true,                 
	captionpos=b,                    
	keepspaces=true,                 
	numbers=left,                    
	numbersep=5pt,                  
	showspaces=false,                
	showstringspaces=false,
	showtabs=false,                  
	tabsize=2
}

\lstset{style=mystyle}

\begin{document}

\title{Конспект седьмого занятия}
\author{Левашов Георгий Игоревич}
\maketitle{}
\section{Функции. Начало}

Для начала, узнаем (или вспомним), что такое функция в математике. Удобно представлять себе функцию как некий "чёрный ящик"\ - мы кладём в него число, а ящик нам по числу выдаёт по определённому правилу другое число. Например, функцией является уже знакомое нам возведение в квадрат --- по числу мы получаем его квадрат.
\\* Давайте теперь узнаем, как нам реализовать возведение во вторую степень на питоне

\begin{lstlisting}[language=Python]
def fun(a):
    return a * a

b = 3
c = fun(b)
print(b, c)
#Output: 3 9
\end{lstlisting}
Создание каждой функции в Python мы начинаем ключевым словом \textcolor{red}{def}. Далее идёт название функции (в моём случае --- лаконичное $fun$), а после в скобках идут параметры через запятую (их также часто называют аргументы функции). Далее идут команды, которые будут выполняться, когда мы будем эту функцию использовать (обратите внимание, отступ!)
\\* Вспоминая аналогию с "чёрным ящиком"\, параметры функции (их может быть много, но об этом позже) --- это то, что мы кидаем в чёрный ящик. А что тогда то, что ящик нам выкидывает? Для того, чтобы функция что-то нам возвращала, используется специальное слово \textcolor{red}{return} (буквально переводится как "возвращать") и после него, собственно, то, что мы хотим вернуть
\\* После того, как мы функцию создали, мы можем её использовать (программисты часто говорят "вызывать") в нашей программе сколько угодно раз. Использовать функцию достаточно просто --- нужно написать её имя и в скобках то, к чему мы это функцию хотим применить (то есть её аргументы) через запятую. В примере я сохранил результат функции в переменную, а потом её вывел.
\\* ВАЖНО: Функция НЕ МЕНЯЕТ значение переданной переменной (b из примера все еще равно 3), потому что в функцию передаётся только ЗНАЧЕНИЕ этой переменной. По этой же причине не может поменять переменную команда $print()$.
\section{Функции посложнее}
Можно считать, что функция - это некоторая подпрограмма, куда мы что-то передали и что-то хотим получить на выходе. То есть, в ней можно использовать всё, что мы изучили и изучим --- циклы и массивы, создавать новые переменные и всё прочее. Поэтому они очень полезны.
\\* Давайте теперь реализуем функцию, которая находит минимум из двух чисел.
\begin{lstlisting}[language=Python]
def min(a, b):
    if (a < b):
        return a
    else:
        return b

print(min(2, 3), end=" ")
a = 4
print(min(3, a), end=" ")
b = 5
print(min(a, b), end=" ")
#Output: 2 3 4

\end{lstlisting}
В данном случае, функции уже передаётся 2 параметра, и она возвращает значение не сразу. Код функции, то есть команды, написанные после $def \ min(a, b):$, называется телом функции. Таким образом, в теле функции мы используем условный оператор \textcolor{red}{if} и с его помощью возвращаем правильный ответ. \\
Как видно в примере, в функции может быть несколько слов \textcolor{red}{return}, но нужно понимать, что \textcolor{red}{return} буквально "возвращается" из функции, так что после него никаких команд выполнено не будет. В частности, может сработать только один \textcolor{red}{return}.
\\* Заметьте, в функцию можно передавать и переменные, и обычные числа.\\
\\
Почему бы функции не возвращать значение типа \textit{Bool}, т.е. "правда" или "ложь" (True/False).
Давайте напишем функцию, которая возвращает True, если число четное и False, если нечетное
\begin{lstlisting}[language=Python]
def even(n):
	if n % 2 == 0:
		return True
	else:
		return False


n = int(input())
if even(n):
	print("Even")
else: 
	print("Odd")

#Input: 4
#Output: Even
#
#Input: 5
#Output: Odd
\end{lstlisting}
Заметьте, что здесь мы вызываем функцию прямо внутри \textcolor{red}{if}-а, она считает значение, выдает True или False, после чего срабатывает \textcolor{red}{if}.

\section{Интересные факты}
Давайте обсудим уже знакомые нам $input()$ и $print()$. До этого момента я преимущественно называл их "командами"\, но на самом деле это самые натуральные функции! Хотя всё ещё нам тяжело понять, что происходит у них "внутри"\, то есть что у них за тело функции, на их примере мы можем ещё несколько интересных вещей про функции сказать
\\* Начнём с $input()$. Заметим, что эта функция ничего не принимает --- вполне допустимо, чтобы у функции не было аргументов, input() они и не нужны. В качестве возвращаемого значения --- считанная строка.
\\* Про print(). У этой функции, наоборот, с аргументами всё в порядке --- это то, что мы хотим вывести. Интересный момент --- в print() мы можем передать и 1, и 2, и 100 строчек, и он их всех выведет. То есть print() поддерживает возможность работы с разным количеством аргументов (мы об этом будем говорить подробнее позже). Остался вопрос --- что же $print()$ возвращает? Давайте узнаем это с помощью кода
\begin{lstlisting}[language=Python]
print(print(2, end=" "))
#Output: 2 None
\end{lstlisting}
Я вывел то, что мне вернул print() как функция! Мне вернулось None --- это специальное слово, которое обозначает "ничего". Можно сказать, что None для Python --- как 0 для обычных чисел. Таким образом, print() возвращает "ничего"\ , мы от него ничего и не ждали.

\section{int() и float()}
\subsection{int()}
Наш старый знакомый \textcolor{red}{int}() - это тоже функция! Он занимается тем, что превращает свой аргумент в целое число. Теперь мы можем понять, что строчка \textcolor{red}{int}(\textcolor{red}{input}()) - это \textit{композиция} двух функций - первая считывает строку, вторая превращает ее в число.\\
Однако превращать в целое число можно не только строковое представление этого числа, но и дробные числа. \textcolor{red}{int}() можно использовать как округление вниз:
\begin{lstlisting}[language=Python]
print(int(3.4), int(3.9), int(4), int(5 / 2))
#Output: 3 3 4 2
\end{lstlisting} 
\subsection{float()}
Раньше мы не сталкивались, но сегодня столкнемся, с задачей считать дробное число. Мы умели считывать только целые числа. Для чтения же дробных есть функция \textcolor{red}{float}(), которая превращает строку-дробное-число в дробное число
\begin{lstlisting}[language=Python]
a = float(input())
print(a ** 2)
#Input: 1.5
#Output: 2.25
\end{lstlisting}
Почему команда называется float() я подробно объяснять не буду. Скажу только, что это происходит от "плавающей" точки в десятичной записи числа. Если интересно узнать поподробнее, можно почитать \href{https://habr.com/ru/post/112953/}{здесь}.

\end{document}