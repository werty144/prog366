\documentclass[12pt]{article} % 12 -- размер шрифта
\usepackage{cmap} % Чтобы можно было копировать русский текст из pdf
\usepackage[T2A]{fontenc}
\usepackage[russian]{babel} % В частности эта строка отвечает за правильные переносы слов в конце строки
\usepackage[utf8]{inputenc} % Проверьте, что кодировка файла -- тоже utf8
\usepackage{amsmath, amssymb} % Чтобы юзать математические символы
\usepackage{ dsfont }
\usepackage{ wasysym }
\usepackage[makeroom]{cancel}
\usepackage{listings}
\usepackage{tcolorbox}

\usepackage[utf8]{inputenc}

\usepackage{listings}
\usepackage{xcolor}
\usepackage{tikz}
\usetikzlibrary {positioning}

\newcommand \tab[1][1cm]{\hspace*{#1}}

\definecolor{codegreen}{rgb}{0,0.6,0}
\definecolor{codegray}{rgb}{0.5,0.5,0.5}
\definecolor{codepurple}{rgb}{0.58,0,0.82}
\definecolor{backcolour}{rgb}{0.95,0.95,0.92}

\lstdefinestyle{mystyle}{
	backgroundcolor=\color{backcolour},   
	commentstyle=\color{codegreen},
	keywordstyle=\color{magenta},
	numberstyle=\tiny\color{codegray},
	stringstyle=\color{codepurple},
	basicstyle=\ttfamily\footnotesize,
	breakatwhitespace=false,         
	breaklines=true,                 
	captionpos=b,                    
	keepspaces=true,                 
	numbers=left,                    
	numbersep=5pt,                  
	showspaces=false,                
	showstringspaces=false,
	showtabs=false,                  
	tabsize=2
}

\lstset{style=mystyle}

\begin{document}

\title{Конспект первого занятия\\
	\large Введение в Python}
\author{Парамонов Антон Игоревич}
\maketitle
\section{Вывод}
\tab Поначалу мы будем использовать компьютер как калькулятор, постепенно совершенствуя его и добавляя новые возможности. Что должен уметь делать любой калькулятор? Считать, разумеется, и \textbf{выводить} ответ, иначе, если мы не можем узнать результат подсчетов, что от него толку. И так, вот как устроен вывод на \textit{python}
\begin{lstlisting}[language=Python]
print(2 + 2)
#Output: 
#4
\end{lstlisting}
\tab Мы просим компьютер напечатать что-либо командой \textcolor{red}{print}, после которой \underline{в круглых скобочках} указываем, что именно мы хотим, чтоб компьютер напечатал.\\
\tab В примере выше и далее, зеленые строчки, начинающиеся с \textcolor{green}{\#} не являются частью программы, с помощью них я демонстрирую вывод (англ. output) программы.\\

\section{Арифметика}
\tab \textit{Python}, как и любой другой язык программирования, умеет выполнять всевозможные арифметические операции, вот набор символов, которыми они обозначаются\\
\begin{tabular}{|l|l|l|l|}
	\hline
	Операция & Значок & Пример & Вывод примера\\
	\hline
	\hline
	Сложение & + & print(2 + 2) & 4\\
	\hline
	Вычитание & - & print(2 - 5) & -3\\
	\hline
	Умножение & * & print(6 * 7) & 42\\
	\hline
	Деление & / & print(3 / 2) &  1.5\\
	\hline
	Возведение в степень & ** & print(5 ** 2) & 25\\
	\hline 
	Скобки & () & print((2 + 2) * 2) & 8\\
	\hline
\end{tabular}\\
\\
\tab Выглядит несложно, но я хотел бы заострить внимание на двух командах
\subsection{Деление}
\tab Возможно, вы заметили, что результат деления 3 на 2 компьютер записал в виде десятичной дроби (с точкой) 1.5. И это разумно, ведь нацело не делится. Однако
\begin{lstlisting}[language=Python]
print(4 / 2)
#Output: 
#2.0
\end{lstlisting}
когда нацело делится, точка все еще остается. Это связано с тем, что \textit{python} заранее не знает, какие числа вы дадите команде /, делящиеся или нет, так что ее результат всегда будет дробным числом, возможно, с нулевой дробной частью.\\
\tab Также нужно понимать, что есть бесконечные дроби. Например, $\frac{1}{3} = 0.3333\ldots$ Компьютер не может вычислить бесконечный хвост, так что \underline{все операции с дробными числами происходят с определенной}\\ \underline{погрешностью}
\begin{lstlisting}[language=Python]
print(1 / 3)
#Output: 
#0.3333333333333333
\end{lstlisting} 
\subsection{Возведение в степень}
\begin{tcolorbox}
	\textit{Математическая справка}\\
	\tab В школе учат, что возведение в степень - это умножить число само на себя сколько-то раз. Например, $5 ^ 3 = \underbrace{5 \cdot 5 \cdot 5}_{3} = 125$. Но тогда что такое $5 ^ {\frac{1}{3}}$? Мы знаем, что у степени есть свойство: $a^p \cdot a^q = a^{p + q}$. Тогда $5^{\frac{1}{3}} \cdot 5^{\frac{1}{3}} \cdot 5^{\frac{1}{3}} = 5 ^ {\frac{1}{3} + \frac{1}{3} + \frac{1}{3} } = 5 ^ 1 = 5$. Т.е. $5 ^ {\frac{1}{3}}$ - такое число, которое умноженное само на себя 3 раза, дает нам 5. Ну так это просто определение кубического корня! $5 ^ {\frac{1}{3}} = \sqrt[3]{5}$. Или более обще: $a ^ {\frac{1}{n}} = \sqrt[n]{a}$.  
\end{tcolorbox}
В возведении в степень после ** можно указывать нецелое число
\begin{lstlisting}[language=Python]
print(36 ** (1/2))
#Output: 
#6.0
\end{lstlisting}
\tab Мы научились извлекать корень любой степени. Результат операции, по тем же соображениям, что и при делении - нецелое число. Отмечу также скобки вокруг 1/2. Если бы их не было, 36 сначала возвелось бы в степень 1, а потом поделилось бы на 2. И результатом было бы 18.0.
\section{Переменные}
\tab Для удобства можно заводить переменные. Подробнее о них поговорим на следующих занятиях, а пока что вот пример использования переменных
\begin{lstlisting}[language=Python]
x = 2
y = 3
print(x)
print(x + y)
print(x * y)
#Output:
#2
#5
#6
\end{lstlisting}
\tab Заметим, что каждая следующая команда \textcolor{red}{print} выводит на новой строчке.
\section{Ошибки}
\tab Ничто не убережет вас от ошибок при написании программы. Если вы совершили оную, то при запуске компьютер сообщит вам об этом в следующем формате
\begin{lstlisting}[language=Python]
print(1 / 0)
#Output:
#Traceback (most recent call last):
#File "<stdin>", line 1, in <module>
#ZeroDivisionError: division by zero
\end{lstlisting}
\tab Сообщения об ошибке поначалу кажутся непонятными и громоздкими. Однако из них можно вычленить полезную информацию. Так например, здесь написано, что ошибка произошла в первой строчке программы (line 1), а именно, было деление на 0 (division by zero). 
\section{Строки}
\tab Команда \textcolor{red}{print} может печатать не только целые и нецелые числа. Она также может печатать строки. Давайте попробуем напечатать букву х
\begin{lstlisting}[language=Python]
print(x)
#Output:
#Traceback (most recent call last):
#File "<stdin>", line 1, in <module>
#NameError: name 'x' is not defined
\end{lstlisting}
\tab Программа упала с ошибкой, но почему?.. Пишет, что не знает, что такое х. Дело в том, что компьютер хочет напечатать, чему равна переменная х, но никакой переменной мы не вводили, вот все и ломается. Если мы хотим напечатать именно букву 'х', ее \underline{нужно заключить в кавычки}.
\begin{lstlisting}[language=Python]
print('x')
#Output:
#x
\end{lstlisting}
Можно и в двойные, без разницы
\begin{lstlisting}[language=Python]
print("I love programming")
#Output:
#I love programming
\end{lstlisting}
\subsection{Операции со строками}
Строки можно складывать 
\begin{lstlisting}[language=Python]
print("Me" + "code")
#Output:
#Mecode
\end{lstlisting}
\tab В таком случае строчки 'припишутся' друг к другу и, \underline{если там не} \underline{было пробела, то он не появится.}\\
\tab Раз мы умеем складывать строки, мы умеем и умножать их на число. Умножить строчку, скажем, на 3 - это сложить ее саму с собой 3 раза.
\begin{lstlisting}[language=Python]
print("Hello! " * 3)
#Output:
#Hello! Hello! Hello!
\end{lstlisting}
\tab Заметим, что после каждого (и после последнего) восклицательного знака есть пробел, поскольку он был в изначальной строчке. 
\end{document}