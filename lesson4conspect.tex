\documentclass[12pt]{article} % 12 -- размер шрифта
\usepackage{cmap} % Чтобы можно было копировать русский текст из pdf
\usepackage[T2A]{fontenc}
\usepackage[russian]{babel} % В частности эта строка отвечает за правильные переносы слов в конце строки
\usepackage[utf8]{inputenc} % Проверьте, что кодировка файла -- тоже utf8
\usepackage{amsmath, amssymb} % Чтобы юзать математические символы
\usepackage{ dsfont }
\usepackage{ wasysym }
\usepackage[makeroom]{cancel}
\usepackage{listings}

\usepackage[utf8]{inputenc}

\usepackage{listings}
\usepackage{xcolor}
\usepackage{tikz}
\usetikzlibrary {positioning}

\definecolor{codegreen}{rgb}{0,0.6,0}
\definecolor{codegray}{rgb}{0.5,0.5,0.5}
\definecolor{codepurple}{rgb}{0.58,0,0.82}
\definecolor{backcolour}{rgb}{0.95,0.95,0.92}

\lstdefinestyle{mystyle}{
	backgroundcolor=\color{backcolour},   
	commentstyle=\color{codegreen},
	keywordstyle=\color{magenta},
	numberstyle=\tiny\color{codegray},
	stringstyle=\color{codepurple},
	basicstyle=\ttfamily\footnotesize,
	breakatwhitespace=false,         
	breaklines=true,                 
	captionpos=b,                    
	keepspaces=true,                 
	numbers=left,                    
	numbersep=5pt,                  
	showspaces=false,                
	showstringspaces=false,
	showtabs=false,                  
	tabsize=2
}

\lstset{style=mystyle}

\begin{document}

\title{Конспект четвертого занятия}
\author{Парамонов Антон Игоревич}
\maketitle
\section{For}
В первом контесте у вас была задача посчитать сумму $\frac{1}{1^2} + \frac{1}{2^2} + \frac{1}{3^2} + \ldots + \frac{1}{10^2} \approx \frac{\pi^2}{6}$. Вычислив такую сумму, несложно посчитать $\pi$. Понятно, что чем больше слагаемых мы выпишем, тем точнее будет приближение. Но выписывать их ручками лень, а если нужно выписать порядка миллиона, то просто невозможно. В таких вот случаях нам на помощь и приходит \textcolor{red}{for} - цикл.\\
Для начала рассмотрим пример попроще и познакомимся с правилами написания цикла
\begin{lstlisting}[language=Python]
for i in range(3): #shapka (golova) cikla
	print("I love") #telo cikla
	print(366) #telo cikla
print("Bye bye!") #not in cycle any more
#Output:
#I love
#366
#I love
#366
#I love
#366
#Bye bye!
\end{lstlisting}
Что здесь произошло. Главная идея цикла - повторение какого-то действия несколько раз. Сколько раз - написано в шапке цикла. В данном случае 3. Какое действие повторять написано в теле цикла, в данном случае мы выводим строчку "I love" и число 366.\\

Что такое написано в шапке, узнаем далее, на настоящий момент надо запомнить, что после некоего слова \textcolor{red}{range} в круглых скобках указано число повторений, после этого стоит \textbf{двоеточие}, а тело цикла пишется с \textbf{отступом}.\\
Теперь разберемся, что это за \textit{i} в шапке цикла. Посмотрите на следующую программу
\begin{lstlisting}[language=Python]
for i in range(5): 
	print("Now i is") 
	print(i)
print("Goodbye!") 
#Output:
#Now i is
#0
#Now i is
#1
#Now i is
#2
#Now i is
#3
#Now i is
#4
#Goodbye!
\end{lstlisting}
\begin{tikzpicture}[node distance = 5 cm, ->, thick]

\node[draw, circle] (A) {Увеличить i};
\node[draw, circle, right of=A] (B) {Выполнить тело};

\path (A) edge [bend left =20] node {} (B);
\path (B) edge [bend left = 20] node {} (A);

\end{tikzpicture}\\
\textit{i} - это такая переменная, которая принимает значения от \textbf{0} до, в нашем случае, 4ех (не 5ти!). И для каждого значения \textit{i} выполнится тело цикла. Т.е. тело цикла выполнится 5 раз, \textit{i} будет принимать значения $\underbrace{0, 1, 2, 3, 4}_{5}$. В этом можно убедиться, посмотрев на вывод программы. На каждом шаге цикла (коих 5) мы просим напечатать строчку "Now i is" и само число \textit{i}. Видно, что с каждым следующим шагом \textit{i} увеличивается на 1, пока не станет равно 4 (т.е. тому числу, которое мы указали в шапке, минус один). \\
Теперь мы можем написать хорошую программу для подсчета суммы $\frac{1}{1^2} + \frac{1}{2^2} + \frac{1}{3^2} + \ldots + \frac{1}{10^2}$. 
\begin{lstlisting}[language=Python]
sum = 0
for i in range(10):
	sum = sum + 1/((i + 1) ** 2)
print(sum)
#Output:
#1.5497677311665408
\end{lstlisting}
Здесь мы завели переменную \textit{sum}, изначально она равна 0, а на каждом шаге цикла мы прибавляем к ней (если быть точнее, мы даем ей новое значение, равное старому плюс что-то) $\frac{1}{(i + 1)^2}$. Следующая табличка демонстрирует пошагово, что тут происходит\\


\begin{tabular}{ |p{3cm}|p{3cm}|p{3cm}|  }

	\hline
	Шаг цикла & i & sum\\
	\hline
	\hline
	До цикла & не существует & 0\\
	\hline
	1&0&$1$\\
	\hline 
	2&1&$1 + \frac{1}{4}$\\
	\hline
	3&2&$1 + \frac{1}{4} + \frac{1}{9}$\\
	\hline
	\vdots & \vdots & \vdots \\
	\hline
	10 & 9 & $\frac{1}{1^2} + \frac{1}{2^2} + \frac{1}{3^2} + \ldots + \frac{1}{10^2}$\\
	\hline
	После цикла & не существует & $\frac{1}{1^2} + \frac{1}{2^2} + \frac{1}{3^2} + \ldots + \frac{1}{10^2}$\\
	\hline
\end{tabular}\\
\\
После цикла у нас есть насчитанная сумма в переменной \textit{sum}, с которой мы вольны делать, что захотим.\\
Заметим, что в коде мы вынуждены писать 'i + 1', потому что \textit{i} принимает значения от 0 до 9, а нам нужно от 1 до 10. Это можно исправить и по-другому:
\begin{lstlisting}[language=Python]
sum = 0
for i in range(1, 11):
	sum = sum + 1/(i ** 2)
print(sum)
#Output:
#1.5497677311665408
\end{lstlisting}
Если в круглых скобках после \textcolor{red}{range} идет два числа, это значит \textit{i} будет принимать значения от первого включительно до второго не включительно (именно поэтому второе число 11, нам ведь нужно до 10).\\
В тех же скобках можно написать и третье число - шаг. А именно то, как меняется наше \textit{i}. Если до этого оно менялось на +1, то теперь можно менять его на любое целое число.\\
\begin{lstlisting}[language=Python]
for i in range(10, 22, 3):
	print(i)
#Output:
#10
#13
#16
#19
\end{lstlisting}
И даже отрицательное (\textit{i} будет уменьшаться):
\begin{lstlisting}[language=Python]
for i in range(15, 10, -1):
	print(i)
#Output:
#15
#14
#13
#12
#11
\end{lstlisting}
\end{document}