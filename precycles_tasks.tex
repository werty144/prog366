\documentclass[12pt]{article} % 12 -- размер шрифта
\usepackage{cmap} % Чтобы можно было копировать русский текст из pdf
\usepackage[T2A]{fontenc}
\usepackage[russian]{babel} % В частности эта строка отвечает за правильные переносы слов в конце строки
\usepackage[utf8]{inputenc} % Проверьте, что кодировка файла -- тоже utf8
\usepackage{amsmath, amssymb} % Чтобы юзать математические символы
\usepackage{ dsfont }
\usepackage{ wasysym }
\usepackage[makeroom]{cancel}
\usepackage{hyperref}
\usepackage{listings}
\usepackage[shortlabels]{enumitem}
\usepackage{fancyhdr}
\usepackage{wasysym}

\begin{document}
	\begin{center}
		\Large{Начальные задачи на for и while}
	\end{center}

Решите каждую из следующих задач, используя \textit{for} и \textit{while}.
\begin{enumerate}
	\item 100 раз напечатайте строку \textit{"Hello!"}
	\item Напечатайте числа от 0 до 100 включительно
	\item Напечатайте числа от 42 до 144 включительно с шагом 3
	\item Посчитайте сумму чисел от 1 до 100
	\item Посчитайте произведение чисел от 1 до 10
	
\end{enumerate}
\end{document}