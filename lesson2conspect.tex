\documentclass[12pt]{article} % 12 -- размер шрифта
\usepackage{cmap} % Чтобы можно было копировать русский текст из pdf
\usepackage[T2A]{fontenc}
\usepackage[russian]{babel} % В частности эта строка отвечает за правильные переносы слов в конце строки
\usepackage[utf8]{inputenc} % Проверьте, что кодировка файла -- тоже utf8
\usepackage{amsmath, amssymb} % Чтобы юзать математические символы
\usepackage{ dsfont }
\usepackage{ wasysym }
\usepackage[makeroom]{cancel}
\usepackage{listings}

\usepackage[utf8]{inputenc}

\usepackage{listings}
\usepackage{xcolor}

\definecolor{codegreen}{rgb}{0,0.6,0}
\definecolor{codegray}{rgb}{0.5,0.5,0.5}
\definecolor{codepurple}{rgb}{0.58,0,0.82}
\definecolor{backcolour}{rgb}{0.95,0.95,0.92}

\lstdefinestyle{mystyle}{
	backgroundcolor=\color{backcolour},   
	commentstyle=\color{codegreen},
	keywordstyle=\color{magenta},
	numberstyle=\tiny\color{codegray},
	stringstyle=\color{codepurple},
	basicstyle=\ttfamily\footnotesize,
	breakatwhitespace=false,         
	breaklines=true,                 
	captionpos=b,                    
	keepspaces=true,                 
	numbers=left,                    
	numbersep=5pt,                  
	showspaces=false,                
	showstringspaces=false,
	showtabs=false,                  
	tabsize=2
}

\lstset{style=mystyle}

\begin{document}

\title{Конспект второго занятия}
\author{Парамонов Антон Игоревич}
\maketitle
\section{Ввод и типы}
На первом занятии мы с вами научились выводить числа и строчки на экран. Сегодня мы изучим ввод. Вот пример программы, которая считывает строчку, и приписывает к ней строчку "FML366".
\begin{lstlisting}[language=Python]
x = input()
print(x + "FML366")
#Input: Anton
#Output: AntonFML366 
\end{lstlisting}
Строчки 3 и 4 с '\#' не являются частью программы, это пример ввода (input) и соответствующего ему вывода (output)\\
Когда вы запустите такую программу, она "повиснет" и будет ждать, пока вы ей не введете что-либо. За это отвечает первая строчка. После того, как вы что-то ввели и нажали Enter, то, что вы ввели, записалось в переменную 'x', которую дальше можно использовать в своих интересах.\\
А вот другая программа
\begin{lstlisting}[language=Python]
x = input()
print(x + 366)
#Input: 7
#Expected output: 373
'''
Actual output: Traceback (most recent call last):
	File "/home/user/PycharmProjects/untitled/test.py", line 2, in <module>
	print(x + 7)
	TypeError: must be str, not int
'''
\end{lstlisting}
Программа падает с ошибкой. Давайте попробуем разобраться, что нам пишет python. Ошибка произошла во второй строчке, ошибка: "must be \textcolor{blue}{str}, not \textcolor{blue}{int}". \\
Дело в том, что в python есть различные \textbf{типы}. Есть тип целое число - \textcolor{blue}{int} (сокращенно от английского integer - целое число), а есть тип строка - \textcolor{blue}{str} (сокращенно от string). И между этими типами можно "переключаться". Давайте посмотрим, как это делать.\\
Вот пример явно неработающей программы:
\begin{lstlisting}[language=Python]
print('3' + 4)
\end{lstlisting}
Действительно, мы хотим прибавить к строке число, это и не должно работать. А вот как это можно исправить:
\begin{lstlisting}[language=Python]
print(int('3') + 4)
#Output: 7
\end{lstlisting}
Команда \textcolor{red}{int} "превращает" строку '3' в число 3. А вот еще способ починить нашу неправильную программу:
\begin{lstlisting}[language=Python]
print('3' + str(4))
#Output: 34
\end{lstlisting} 
Здесь мы с помощью команды \textcolor{red}{str} превратили число 4 в строку '4', после чего сложили две строки.\\
Отлично, теперь мы можем исправить программу, которая прибавляет 366
\begin{lstlisting}[language=Python]
x = input()
print(int(x) + 366)
#Input: 7
#Output: 373
\end{lstlisting}
Ну а если вы хотите сразу сделать так, чтоб 'x' был числом, а не применять к нему каждый раз команду \textcolor{red}{int}, можно написать так
\begin{lstlisting}[language=Python]
x = int(input())
print(x + 366)
#Input: 7
#Output: 373
\end{lstlisting}
Здесь мы ввели строчку, сразу превратили ее в число и записали значение (уже числовое) в переменную 'x', которая теперь это значение хранит. 

\section{Частное и остаток}
А вот вам еще две команды.\\
Первая команда '\%' - остаток от деления
\begin{lstlisting}[language=Python]
print(14 % 3)
#Output: 2
\end{lstlisting}
Вторая: '//' - частное от деления
\begin{lstlisting}[language=Python]
print(14 // 3)
#Output: 4
\end{lstlisting}
Пользуйтесь на здоровье и решайте контесты!
\end{document}