\documentclass[12pt]{article} % 12 -- размер шрифта
\usepackage{cmap} % Чтобы можно было копировать русский текст из pdf
\usepackage[T2A]{fontenc}
\usepackage[russian]{babel} % В частности эта строка отвечает за правильные переносы слов в конце строки
\usepackage[utf8]{inputenc} % Проверьте, что кодировка файла -- тоже utf8
\usepackage{amsmath, amssymb} % Чтобы юзать математические символы
\usepackage{ dsfont }
\usepackage{ wasysym }
\usepackage[makeroom]{cancel}
\usepackage{hyperref}
\usepackage{listings}
\usepackage[shortlabels]{enumitem}
\usepackage{fancyhdr}

\hypersetup{
	colorlinks=true,       % false: boxed links; true: colored links
	linkcolor=blue,        % color of internal links
	citecolor=blue,        % color of links to bibliography
	filecolor=magenta,     % color of file links
	urlcolor=blue
}

\pagestyle{fancy}
\fancyhf{}
\rhead{Последовательности}
\lhead{Динамическое программирование}
\rfoot{Page \thepage}

% \pgfplotsset{compat=1.16}
\begin{document}
	
	\begin{center}
		{\Large\bf
			Домашнее задание
		}
	\end{center}
	\vspace*{-1em}\noindent \underline{\hbox to 1\textwidth{{ } \hfil{ } \hfil{ } }}
	\section*{Напоминание}
	Числовая последовательность -- последовательность чисел, например\\ $1,\ 3,\ 5\ldots$ -- нечетные числа. i-ый член последовательности будем обозначать $a_i$. Будем говорить, что нам задали последовательность, если мы можем сказать, чему равен любой член последовательности. Если последовательность конечна, то задать ее легко, можно просто перечислить все ее члены. Чтобы задать бесконечную последовательность, существует два способа
	\begin{enumerate}
		\item Описать, как по номеру элемента получить его значение. Например, $i$-ое нечетное число = $2i - 1$. 
		\item Описать, как по предыдущим членам последовательности получить следующий. Например, чтоб получить следующее нечетное число, надо к предыдущему прибавить 2, т.е. $a_i = a_{i - 1} + 2$.\\
		Заметим, однако, что для первого элемента не существует предыдущего, так что его нужно задать явно, т.е. в случае с нечетными числами в добавок к формуле выше нужно указать, что $a_1 = 1$.\\
		Таким образом, чтобы задать последовательность этим способом, надо указать две вещи:\\
		\textbf{База:} значения некоторого количества начальных элементов последовательности.\\
		\textbf{Переход:} правило, описывающее, как, зная предыдущие элементы, получить следующий.\\		
	\end{enumerate}
	Больше примеров, иллюстрирующих эти два способа, есть в  \href{https://informatics.msk.ru/file.php/3097/lesson6conspect.pdf}{конспекте}. 
	
	\newpage
	\section*{Задачи}
	\begin{enumerate}
		\item По первым нескольким членам догадайся, по какому правилу формируется последовательность, и опиши ее первым и вторым способами
		\begin{enumerate}
			\item $7,\ 7,\ 7,\ 7\ldots$
			\item $1,\ 4,\ 7,\ 10\ldots$
			\item $2,\ 4,\ 8,\ 16\ldots$
			\item $2,\ 6,\ 18,\ 54\ldots$
			\item $1,\ 8,\ 27,\ 64\ldots$ (здесь будет комбинация двух способов)
			\item $2,\ 3,\ 6,\ 18,\ 108\ldots$ (только вторым способом)
			\item $1,\ 5,\ 6,\ 11,\ 17\ldots$ (только вторым способом)
		\end{enumerate}
		\item Вася загадал последовательность, записал ее вторым способом и послал другу. Однозначно ли у Васи получилось задать? Если нет, объясни почему. Если под описание подходит несколько последовательностей, укажи хотя бы 2. Если последовательность задана однозначно, выпиши первые 10 ее членов. Вот описания Васи
		\begin{enumerate}
			\item $a_i = a_{i - 1} + 2$
			\item $a_1 = 5\ \vline \ a_i = a_{i - 1} \cdot a_{i - 2}$
			\item $a_1 = 2,\ a_2 = 3\ \vline \ a_i = a_{i-1} + a_{i-2}$
			\item $a_1 = 2,\ a_2 = 3\ \vline \ a_i = a_{i-1}^2$
			\item $a_1 = 7,\ a_2 = -2\ \vline \ a_i = a_{i - 1} + 2a_{i - 2} + 3a_{i - 3}$ 
			\item $a_1 = 1,\ a_2 = 3\ \vline\ a_i = a_{i - 2}$
		\end{enumerate} 
		\newpage
		\item Заполни таблицу, если значение в клетке равно сумме значений в соседях слева и сверху. Если нет соседа слева, значение равно значению в соседе сверху. Если нет соседа сверху, значение равно значению в соседе слева.\\
		\\
		\begin{tabular}{|c|c|c|c|}
			\hline
			1 & . & . & . \\
			\hline 
			. & . & . & . \\
			\hline
				. & . & . & . \\
			\hline
				. & . & . & . \\
			\hline
		\end{tabular}
		\\ \\
		Обозначим значение в клетке на пересечении $i$-ого столбца и $j$-ой строки за $a_{ij}$. Придумай формулу для $a_{ij}$. 
	\end{enumerate}
	

\end{document}