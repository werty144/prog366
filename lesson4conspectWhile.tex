\documentclass[12pt]{article} % 12 -- размер шрифта
\usepackage{cmap} % Чтобы можно было копировать русский текст из pdf
\usepackage[T2A]{fontenc}
\usepackage[russian]{babel} % В частности эта строка отвечает за правильные переносы слов в конце строки
\usepackage[utf8]{inputenc} % Проверьте, что кодировка файла -- тоже utf8
\usepackage{amsmath, amssymb} % Чтобы юзать математические символы
\usepackage{ dsfont }
\usepackage{ wasysym }
\usepackage[makeroom]{cancel}
\usepackage{listings}
\usepackage{tcolorbox}

\usepackage[utf8]{inputenc}

\usepackage{listings}
\usepackage{xcolor}
\usepackage{tikz}
\usetikzlibrary {positioning}

\newcommand \tab[1][1cm]{\hspace*{#1}}

\definecolor{codegreen}{rgb}{0,0.6,0}
\definecolor{codegray}{rgb}{0.5,0.5,0.5}
\definecolor{codepurple}{rgb}{0.58,0,0.82}
\definecolor{backcolour}{rgb}{0.95,0.95,0.92}

\lstdefinestyle{mystyle}{
	backgroundcolor=\color{backcolour},   
	commentstyle=\color{codegreen},
	keywordstyle=\color{magenta},
	numberstyle=\tiny\color{codegray},
	stringstyle=\color{codepurple},
	basicstyle=\ttfamily\footnotesize,
	breakatwhitespace=false,         
	breaklines=true,                 
	captionpos=b,                    
	keepspaces=true,                 
	numbers=left,                    
	numbersep=5pt,                  
	showspaces=false,                
	showstringspaces=false,
	showtabs=false,                  
	tabsize=2
}

\lstset{style=mystyle}

\begin{document}
\title{Конспект четвертого занятия\\
	\large Цикл с помощью While}
\author{Парамонов Антон Игоревич}
\maketitle
\section{While}
\tab Мы научились повторять какое-то действие многократно с помощью команды \textcolor{red}{for}, однако эта "многократность" ограничена. Напомню, что нам нужно указать диапазон и шаг, с которым переменная \textit{i} будет перемещаться по диапазону. Когда \textit{i} пройдет весь диапазон, цикл завершится. У этого механизма есть две слабости:
\begin{enumerate}
	\item Мы должны указать диапазон заранее и не можем его менять в зависимости от того, что в цикле происходит
	\item Переменная \textit{i} меняется с каким-то постоянным шагом. А нам, возможно, понадобится изменять ее в зависимости от того, что происходит в цикле, или, например, все время умножать на 3, а не прибавлять какое-то фиксированное число
\end{enumerate}
\tab Возможно, объяснение недостатков \textcolor{red}{for} показалось вам абстрактным, давайте разбираться на примере. Вот как пишется \textcolor{red}{while}
\begin{lstlisting}[language=Python]
i = 0
while (i < 3):
	print("Now i is")
	print(i)
	i = i + 1
print("While is over")
#Output:
#Now i is
#0
#Now i is
#1
#Now i is
#2
#While is over
\end{lstlisting}
\tab Это программа делает точно то же самое, что 
\begin{lstlisting}[language=Python]
for i in range(3):
	print("Now i is")
	print(i)
print("While is over")
\end{lstlisting}
\tab Давайте разбираться в коде. Во-первых, заметим, что переменную \textit{i} нужно завести самостоятельно (1ая строчка) и указывать, что с ней делать и в какой момент цикла, нужно тоже самостоятельно, так в 5ой строчке мы прибавляем к \textit{i} 1.\\
\tab \underline{В круглых скобках} после команды \textcolor{red}{while} мы пишем условие. Пока это условие выполняется, цикл будет выполняться. \underline{После круглых скобок} \underline{ставится двоеточие}. \underline{Тело цикла пишется с отступом.}\\
\tab Большие возможности - большая ответственность. While нужно писать аккуратно. Заметим, что \underline{само по себе \textit{i} не менятся}, что с ним делать и когда это делать, программист указывает самостоятельно. В нашей программе можно было бы забыть увеличивать \textit{i} на 1, тогда программа ушла бы в бесконечный цикл
\begin{lstlisting}[language=Python]
i = 0
while (i < 3):
	print("Now i is")
	print(i)
print("While is over")
#Output:
#Now i is
#0
#Now i is
#0
#Now i is
#0
#Now i is
#0
#.
#.
#.
\end{lstlisting}
\tab Переменная \textit{i} никак не меняется, поэтому условие $\textit{i} < 3$ выполняется из раза в раз, и программа бесконечно выводит одно и то же.\\
\tab Хорошо, мы научились с помощью \textcolor{red}{while} симулировать \textcolor{red}{for}, давайте теперь сделаем то, что с помощью \textcolor{red}{for} у нас сделать бы не получилось. Будем менять \textit{i} не прибавлением какого-то числа. Допустим мы хотим вывести все степени тройки от 2 до 100. Пожалуйста:
\begin{lstlisting}[language=Python]
i = 3
while (i < 100):
	print(i)
	i = i * 3
#Output:
#3
#9
#27
#81
\end{lstlisting}
\tab Или перед нами такая задача: Нам вводят целые числа, при этом сколько их будет, заранее не известно, однако нам сказали, что последовательность закончится нулем. Наша задача - сказать сумму всех введенных чисел.
\begin{lstlisting}[language=Python]
summ = 0
a = 1
while (a != 0):
	a = int(input())
	summ = summ + a
print("Sum is", a)
#Input:
#1
#2
#-3
#5
#7
#0
#Output:
#Sum is 12
\end{lstlisting}
\tab Здесь мы завели переменную \textit{a}, в которую будем записывать числа, которые нам вводят, и переменную \textit{summ}, в которой будем насчитывать сумму. Изначально сумма 0, а вот \textit{a} можно  назначить любое значение, все равно мы его "затрем"  считав число в 4ой строчке и записав его в \textit{a}. Главное, не сделать \textit{a} нулем сразу, иначе в цикл мы попросту не попадем, т.к. условие не выполнится.\\
\tab Раньше мы этого не обсуждали, но в скобочках у \textcolor{red}{print} можно указывать несколько вещей через запятую, тогда они напечатаются через пробел.
\end{document}